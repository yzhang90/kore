%%%%%%%%%% PACKAGES
\usepackage{comment}
\usepackage{booktabs}                           
%\usepackage{subcaption} 
\usepackage{mathtools}
\usepackage{amsmath}
\usepackage{amssymb}
\usepackage{amsfonts}
\usepackage{prftree}
\usepackage{newtxtext}
\usepackage{newtxmath}
\usepackage[bbgreekl]{mathbbol} % must go after newtxtext & newtxmath
\usepackage{tikz}\usetikzlibrary{arrows}\usepgflibrary{arrows}
\usepackage{tabularx,multirow}
\usepackage{xspace}
\usepackage{wrapfig}
\usepackage{array}
\usepackage{todonotes}
\usepackage[normalem]{ulem}
\usepackage{multicol}
\usepackage{thmtools} 
\usepackage{thm-restate}
\usepackage{stmaryrd}
\usepackage{url}
\usepackage[hidelinks]{hyperref}
\usepackage{listings}% http://ctan.org/pkg/listings
\lstset{
	basicstyle=\ttfamily,
	mathescape
}
\usepackage{algorithm2e}
\usepackage{enumitem}
\usepackage[nocompress]{cite}

% for dealing with underscores
\usepackage[T1]{fontenc}

%%%%%%%%%% UNCLASSIFIED
% text
\newcommand{\BMC}{Bounded Model Checking\xspace}
\newcommand{\MmuL}{Matching $\mu$-Logic\xspace}
\newcommand{\mmul}{matching $\mu$-logic\xspace}
\newcommand{\ml}{matching logic\xspace}
\newcommand{\ModmuL}{Modal $\mu$-Logic\xspace}
\newcommand{\modmul}{modal $\mu$-logic\xspace}
\newcommand{\PS}{\mathcal{H}}
\newcommand{\nats}{\mathbb{N}} % set of natural numbers
\makeatletter % Roman numerals
\newcommand{\rmnum}[1]{\romannumeral #1}
\newcommand{\Rmnum}[1]{\expandafter\@slowromancap\romannumeral #1@}
\newcommand*{\rom}[1]{\expandafter\@slowromancap\romannumeral #1@}
\makeatother
\newcommand{\K}{$\mathbb{K}$\xspace} % \K framework
\newcommand{\doubleslash}{//\xspace}
\newcommand{\LTL}{\textsf{LTL}\xspace}
\newcommand{\CTL}{\textsf{CTL}\xspace}

% basic math
\newcommand{\pset}[1]{\mathcal{P}(#1)} % powerset
\newcommand{\cln}{\mathord{:}}
\newcommand{\ldot}{\mathord{.}}
\newcommand{\imp}{\to}
\newcommand{\dimp}{\leftrightarrow}
\newcommand{\setsymdiff}{\mathbin{\triangle}}
\newcommand{\FV}{\mathit{FV}}
\newcommand{\To}{\Rightarrow}

% an implementation of widebar
% obtained from [[https://tex.stackexchange.com/questions/16337/
% can-i-get-a-widebar-without-using-the-mathabx-package]]
\makeatletter
\let\save@mathaccent\mathaccent
\newcommand*\if@single[3]{%
	\setbox0\hbox{${\mathaccent"0362{#1}}^H$}%
	\setbox2\hbox{${\mathaccent"0362{\kern0pt#1}}^H$}%
	\ifdim\ht0=\ht2 #3\else #2\fi
}
%The bar will be moved to the right by a half of \macc@kerna, which is computed 
%by amsmath:
\newcommand*\rel@kern[1]{\kern#1\dimexpr\macc@kerna}
%If there's a superscript following the bar, then no negative kern may follow 
%the bar;
%an additional {} makes sure that the superscript is high enough in this case:
\newcommand*\widebar[1]{\@ifnextchar^{{\wide@bar{#1}{0}}}{\wide@bar{#1}{1}}}
%Use a separate algorithm for single symbols:
\newcommand*\wide@bar[2]{\if@single{#1}{\wide@bar@{#1}{#2}{1}}{\wide@bar@{#1}{#2}{2}}}
\newcommand*\wide@bar@[3]{%
	\begingroup
	\def\mathaccent##1##2{%
		%Enable nesting of accents:
		\let\mathaccent\save@mathaccent
		%If there's more than a single symbol, use the first character instead 
		%(see below):
		\if#32 \let\macc@nucleus\first@char \fi
		%Determine the italic correction:
		\setbox\z@\hbox{$\macc@style{\macc@nucleus}_{}$}%
		\setbox\tw@\hbox{$\macc@style{\macc@nucleus}{}_{}$}%
		\dimen@\wd\tw@
		\advance\dimen@-\wd\z@
		%Now \dimen@ is the italic correction of the symbol.
		\divide\dimen@ 3
		\@tempdima\wd\tw@
		\advance\@tempdima-\scriptspace
		%Now \@tempdima is the width of the symbol.
		\divide\@tempdima 10
		\advance\dimen@-\@tempdima
		%Now \dimen@ = (italic correction / 3) - (Breite / 10)
		\ifdim\dimen@>\z@ \dimen@0pt\fi
		%The bar will be shortened in the case \dimen@<0 !
		\rel@kern{0.6}\kern-\dimen@
		\if#31
		\overline{\rel@kern{-0.6}\kern\dimen@\macc@nucleus\rel@kern{0.4}\kern\dimen@}%
		\advance\dimen@0.4\dimexpr\macc@kerna
		%Place the combined final kern (-\dimen@) if it is >0 or if a 
		%superscript follows:
		\let\final@kern#2%
		\ifdim\dimen@<\z@ \let\final@kern1\fi
		\if\final@kern1 \kern-\dimen@\fi
		\else
		\overline{\rel@kern{-0.6}\kern\dimen@#1}%
		\fi
	}%
	\macc@depth\@ne
	\let\math@bgroup\@empty \let\math@egroup\macc@set@skewchar
	\mathsurround\z@ \frozen@everymath{\mathgroup\macc@group\relax}%
	\macc@set@skewchar\relax
	\let\mathaccentV\macc@nested@a
	%The following initialises \macc@kerna and calls \mathaccent:
	\if#31
	\macc@nested@a\relax111{#1}%
	\else
	%If the argument consists of more than one symbol, and if the first token is
	%a letter, use that letter for the computations:
	\def\gobble@till@marker##1\endmarker{}%
	\futurelet\first@char\gobble@till@marker#1\endmarker
	\ifcat\noexpand\first@char A\else
	\def\first@char{}%
	\fi
	\macc@nested@a\relax111{\first@char}%
	\fi
	\endgroup
}
\makeatother

% mathsc fonts
\newcommand{\mathsc}[1]{{\normalfont\textsc{#1}}} % small caps in math mode
\newcommand{\Var}{\mathsc{Var}}
\newcommand{\EVar}{\mathsc{EVar}}
\newcommand{\SVar}{\mathsc{SVar}}
\newcommand{\PVar}{\mathsc{PVar}}
\newcommand{\Pattern}{\mathsc{Pattern}}

% mathsf/textsf fonts
\newcommand{\TS}{\textnormal{\textsf{TS}\xspace}}
\newcommand{\RS}{\textnormal{\textsf{RS}\xspace}}
\newcommand{\Lang}{\textnormal{\textsf{Lang}\xspace}}
\newcommand{\cfg}{\textnormal{\textsf{cfg}\xspace}}
\newcommand{\init}{\textnormal{\textsf{init}\xspace}}
\newcommand{\prop}{\textnormal{\textsf{prop}\xspace}}
\newcommand{\pattern}{\textnormal{\textsf{pattern}\xspace}}
\newcommand{\predicate}{\textnormal{\textsf{predicate}\xspace}}
\newcommand{\sub}{\textnormal{\textsf{sub}\xspace}}
\newcommand{\IMP}{{\mathsf{IMP}}}

% mathcal font
\newcommand{\Acal}{\mathcal{A}}
\newcommand{\Ccal}{\mathcal{C}}
\newcommand{\Fcal}{\mathcal{F}}
\newcommand{\Qcal}{\mathcal{Q}}

% mathbb font
\newcommand{\Sbb}{\mathbb{S}}

% sorts using mathit fonts
\newcommand{\Cfg}{\mathit{Cfg}}
\newcommand{\State}{\textit{State}}

% symbol
\newcommand{\rhobar}{\bar{\rho}}
\newcommand{\sig}{\mathbb{\Sigma}}
\newcommand{\sigmaM}{\sigma_M}
\DeclarePairedDelimiter{\ceil}{\lceil}{\rceil}
\DeclarePairedDelimiter{\floor}{\lfloor}{\rfloor}
\DeclarePairedDelimiter{\bracket}{\llbracket}{\rrbracket}

% contexts
\newcommand{\CSub}[1]{C_{#1}}
\newcommand{\Csigma}{\CSub{\sigma}}
\newcommand{\Csigmaapp}[1]{\CSub{\sigma}[#1]}

% name of the proof rules
\newcommand{\prule}[1]{\textnormal{\textsc{(#1)}}}

\newcommand{\modusponens}{\prule{Modus Ponens}\xspace}
\newcommand{\propagationbottom}{\prule{Propagation$_\bot$}\xspace}
\newcommand{\propagationvee}{\prule{Propagation$_\vee$}\xspace}
\newcommand{\framing}{\prule{Framing}\xspace}
\newcommand{\existence}{\prule{Existence}\xspace}
\newcommand{\singletonvariable}{\prule{Singleton Variable}\xspace}
\newcommand{\step}{\prule{STEP}\xspace}

% modalities
\newcommand{\wnext}{{\circ}}
\newcommand{\snext}{{\bullet}}
\newcommand{\eventually}{{\lozenge}}
\newcommand{\always}{{\square}}
\newcommand{\until}{\mathbin{U}}
\newcommand{\wellfounded}{{\mathsf{WF}}}
